\documentclass[onecolumn]{mysimple}
\title{Exploration of New Zealand and Auckland categorised criminal-conviction statistics between 1980–2014}
\author{Alan Heays}
\begin{document}
\maketitle

\section*{Main points}
\begin{itemize}
  \item Numerous statistically-significant trends and anomalies are present in the categorised data.
  \item Conviction rates and population criminality are decreasing.
  \item Conviction rates within Auckland and elsewhere are similar for most statistics.
  \item A hypothesis of increased traffic-offence enforcement correlated with subsequently reduced convictions in multiple categories.
  \item A hypothesis of societal or enforcement policy change leading to spikes in convictions for threatening behaviour, particularly in Auckland
\end{itemize}

\section*{Introduction and data analysis}

The purpose of this document is to identify hypotheses and insights by  the analysis of quantitative crime statistics spanning the years 1980 to 2014.

The available data includes annual criminal conviction frequency, and is broken down according to the principal offence categories of the ANZSOC standard. \footnote{\url{https://www.abs.gov.au/statistics/classifications/australian-and-new-zealand-standard-offence-classification-anzsoc}}
National frequencies (``Total Regions'') are given along with a subset attributed to an ``Auckland Cluster''.  
I used the difference of these two to compute a third frequency, ``Total Regions Ex Auckland'', so a clearer binary comparison can be drawn from data distinguishing Auckland crime from everywhere else in the country.
A visualisation of the conviction frequency for all offence categories and regions is given as an Appendix ref ???.

Trends in the total crime frequency informs resourcing of the enforcement and court system, but can be usefully supplemented by a rate-per-population that levels out the effect of population growth.
This permits the analysis of changes in the level of criminality.
An annual estimate of New Zealand's population \footnote{Obtained from \url{https://figure.nz/chart/MFMkVhvbhuVFbiWr}. This data source references Stats NZ (\url{https://infoshare.stats.govt.nz/}) but the full time-period data set is not apparently currently available.}  is used to normalise the national crime rate.

The population falling within the boundaries of the loosely defined ``Auckland Cluster'' does not appear to correspond to the Auckland Regional Council.
Because, normalising ``Auckland Cluster'' conviction frequency by the regional population obtained form Stats NZ finds a implausibly lower crime rate per person with Auckland than elsewhere.
The jurisdiction boundary of the ``Auckland Cluster'' should be clarified in further analysis of this data.
In the meantime, the ratio of ``Total Regions'' and ``Auckland Cluster'' all-crime frequency is used as a proxy of their population, implying that 19\% of the national population is located in the cluster, and 81\% in ``Total Regions Ex Auckland''.
Adopting these percentages for all years in the data set the population-normalised crime frequencies of all data was computed, and visualisations attached as an appendix ref???.

The meaningfulness of some year-to-year variations in some crime categories is doubtful due to the rarity of some offences and statistical randomness.
An approximate statistical uncertainty is computed to help gauge the significance time variations in the data set.\footnote{Assuming Poisson statistics controls the distribution of data the standard deviation convocation rate is computed according to its square-root.  
Two-standard-deviation error bars are used in plots including the statistical uncertainty.}

\section*{Trends in the data}

\subsection*{Convictions falling}
\begin{figure}
  \centering
  \includegraphics{figures/total_offenses.pdf}
  \caption{Total criminal convictions.}
  \label{fig:total crime}
\end{figure}

The baseline number of annual convictions, plotted in Fig.~\ref{fig:total crime}, is decreasing consistently across the data time period, but with significant spikes.
The annual national total has decreased ???\% comparing 1985 with 2014.
The number of convictions per 100$\,$000 people (Fig.~\ref{fig:total crime}) has been falling even faster due to population growth, decreasing ???\% between these two years.
The normalisation to population also demonstrates the similarity of total conviction rates in Auckland and elsewhere.

\begin{figure}
  \centering
  \includegraphics{figures/total_offenses_except_traffic.pdf}
  \caption{Total criminal convictions excluding ``Traffic and vehicle regulatory offences''.}
  \label{fig:total crime}
\end{figure}

The anomalous conviction spikes centred on 1981 and 1989 are attributable to the most frequent conviction category of ``Traffic and vehicle regulatory offences''.
This includes most vehicle, bicycle, and pedestrian offences but not those leading to injury or driving under the influence.
Excluding this category from the total conviction frequency clarifies the steady decrease in the total frequency of the remaining data, including a more rapid improvement in the Auckland cluster than elsewhere. 

\section*{Anomalies in the data}

\subsection*{Reliability of early data}

The reliability of 1980 and 1981 tallies in this data set should be investigated because of anomalously high rates in several categories: ``Traffic and vehicle regulatory offences'',Miscellaneous Offences''.
These years precede the establishment of the ANZSOC standard in 1997, and the re-categorisation of this historical data may suffer from quality issues. 
The three spurious categories also comprise a broad range of relatively minor offences that may baffle classification.

\subsection*{Hypothesis: Convictions for traffic offences and road injuries}

\begin{figure}
  \centering
  \includegraphics{figures/traffic_stuff.pdf}
  \caption{}
  \label{fig:traffic stuff}
\end{figure}


The frequency of convictions in the ``Traffic and vehicle regulatory offences'' category is steady in time and between regions, apart from the unusually high rates in 1981 and 1989. 
The doubled rate of such convictions occurring in 1988 and 1989 relative to adjacent time periods correlates with the peak conviction rate in two other categories, ``Dangerous Or Negligent Acts Endangering Persons'' and ``Homicide And Related Offences'', followed by long-term decreasing rates.
These are plotted in Fig.~\ref{fig:total crime}, where error bars based on the statistical uncertainty of rare ``Homicide And Related Offences'' offences indicates where small fluctuations be ignored.
A constituent of the ``Dangerous Or Negligent Acts Endangering Persons'' conviction total is due to traffic-related injuries and driving under the influences, while ``Homicide And Related Offences'' includes traffic deaths.  

I propose investigating whether a legislative or traffic-offence enforcement policy change initiated around 1988 is attributable a spike in regulatory offence convictions and reduction in the occurrence of dangerous or fatal traffic behaviour.
This could be served by obtaining conviction data set broken down into ANZSOC subcategories separating traffic offences, and researching the occurrence of legislation or policy initiatives at that time. \footnote{A brief public search suggests that drunk driving and speed-camera enforcement campaigns occurred somewhat later.}


 
 - some roughish categories going upwards, not as bad in Auckland. Other cities?

 - ``Homicide And Related Offences'' and ``Dangerous Or Negligent Acts Endangering Persons'', connected to ``Traffic And Vehicle Regulatory Offences''?


\subsection*{Acts Intended To Cause Injury}

\begin{figure}
  \centering
  \includegraphics{figures/injury_stuff.pdf}
  \caption{}
  \label{fig:injury stuff}
\end{figure}

Similarly anomalous conviction rates occur in both of the categories ``Acts Intended To Cause Injury'' and ``Abduction, Harassment And Other Offences Against The Person''.
Both categories include offences involving threatening behaviour (distinguished by being face-to-face or remove) but are otherwise apparently distinct.
The time series of these data are plotted in Fig.~\ref{fig:injury stuff} and feature a doubling of convictions in over the two years 1993 and 1994, and a large increase convictions in the Auckland cluster beginning around 2000.

Several hypotheses seem plausible.  
The step-change in frequency occurring in 1993/1994 may result in a crime-classification policy of this historical data that precedes the ANZSOC standard.
Alternatively, a legislative or enforcement policy change the bringing of charges in these categories might be identifiable.
A change in the true crime rate of this magnitude over such a short time span seems less likely.
The Auckland-only conviction increase after 2000 is unlikely to be a classification anomaly in the collected data, and suggests either an enforcement or charge policy difference in the Auckland police and court system, or a genuinely more threatening Auckland culture.

Additional data is needed to unpick the origins of these anomalies.
An extended dataset included ANZSOC subcategories should be obtained, and regional data that isolates metropolitan regions aside from Auckland may provide further sociological indicators.



\section*{Conclusion}

\section*{Appendix}

\begin{figure}
  \centering
  \caption{ddd}
  \includegraphics{figures/all_offenses_0.pdf}
\end{figure}

\begin{figure}
  \centering
  \caption{ddd}
  \includegraphics{figures/all_offenses_1.pdf}
\end{figure}

\end{document}