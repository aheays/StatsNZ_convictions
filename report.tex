\documentclass[onecolumn]{mysimple}
\title{Exploration of New Zealand and Auckland categorised crime statistics 1981–2014}
\author{Alan Heays}
\begin{document}
\maketitle

\section*{Main points}
\begin{itemize}
  \item Numerous statistically-significant trends and anomalies are present in the broadly categorised data.
  \item \dots
  \item ``Dig out Auckland Cluster''
  \item ANZSOC subcategories
  \item Other cities clusters
\end{itemize}

\section*{Introduction and data analysis}

The purpose of this document is to identify hypotheses and insights by  the analysis of quantitative crime statistics spanning the years 1981 to 2014.

The supplied annual conviction frequency is broken down according to the major offence categories of the ANZSOC standard. \footnote{\url{https://www.abs.gov.au/statistics/classifications/australian-and-new-zealand-standard-offence-classification-anzsoc}}
National frequencies (``Total Regions'') are given along with a subset attributed to an ``Auckland Cluster''.  
I used the difference of these two to compute a third frequency, ``Total Regions Ex Auckland'', so a clearer binary comparison can be drawn from data distinguishing Auckland crime from everywhere else in the country.
A visualisation of the conviction frequency for all offence categories and regions is given as an Appendix ref ???.

Trends in the total crime frequency informs resourcing of the enforcement and court system, but can be usefully supplemented by a rate-per-population that levels out the effect of population growth.
This permits the analysis of changes in the level of criminality.
An annual estimate of New Zealand's population \footnote{Obtained from \url{https://figure.nz/chart/MFMkVhvbhuVFbiWr}. This data source references Stats NZ (\url{https://infoshare.stats.govt.nz/}) but the full time-period data set is not apparently currently available.}  is used to normalise the national crime rate.

The population falling within the boundaries of the loosely defined ``Auckland Cluster'' does not appear to correspond to the Auckland Regional Council.
Because, normalising ``Auckland Cluster'' conviction frequency by the regional population obtained form Stats NZ finds a implausibly lower crime rate per person with Auckland than elsewhere.
The jurisdiction boundary of the ``Auckland Cluster'' should be clarified in further analysis of this data.
In the meantime, the ratio of ``Total Regions'' and ``Auckland Cluster'' all-crime frequency is used as a proxy of their population, implying that 19\% of the national population is located in the cluster, and 81\% in ``Total Regions Ex Auckland''.
Adopting these percentages for all years in the data set the population-normalised crime frequencies of all data was computed, and visualisations attached as an appendix ref???.

\section*{Data source and analysis}

 - description of data

 
\section*{Trends in the data}

\begin{figure}
  \includegraphics{figures/total_offenses.pdf}
\end{figure}

 - overall crime down
 
 - smooth curve if remove traffic
 
 - some roughish categories going upwards, not as bad in Auckland. Other cities?

 - ``Homicide And Related Offences'' and ``Dangerous Or Negligent Acts Endangering Persons'', connected to ``Traffic And Vehicle Regulatory Offences''?

\section*{Anomalies in the data}

\subsection*{Traffic And Vehicle Regulatory Offences}
\subsection*{Acts Intended To Cause Injury}
\subsection*{Others}

\section*{Conclusion}

\section*{Appendix}

\begin{figure}
  \centering
  \caption{ddd}
  \includegraphics{figures/all_offenses_0.pdf}
\end{figure}

\begin{figure}
  \caption{ddd}
  \centering
  \includegraphics{figures/all_offenses_1.pdf}
\end{figure}

\end{document}